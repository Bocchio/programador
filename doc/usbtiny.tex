\documentclass[a4paper,10pt]{article}


%opening
\title{Uso del usbtiny}
\author{Gabriel Gavinowich}

\begin{document}

\maketitle
\tableofcontents

\begin{abstract}
Este documento describe como usar el usbtiny para programar micros de la familia 8051.
\end{abstract}

\section{Tramas}


Trama POWERUP:

wValue = El periodo del sck en micro segundos
wIndex = Se toma la parte baja:  si el reset es por bajo o por alto:   0 => reset en bajo
					       !=0  => reset alto

En nuestro caso, el reset tiene que ser alto 
El sck, todavia no sabemos, ponelo en 10 que funciona con nuestros cristales. Luego pensamos como sacarlo.

La trama que manda el avrdude:

	usb_control( USBTINY_POWERUP, sck_period, resetState );

	
Trama de Configure:

wValue:  Se divide en parte baja y alta
	wValue low: El byte de status

	//ximmmTTT	TTT: cantidad de bytes a enviar por spi. 100: S8253, S52 y AVR tienen 4 bytes. 
								 011: S8252 S53 tienen 3 bytes.
			mmm: tipo del micro		000	Avr
							001     S8253 (por ahora el S52 es igual al S8253, pero tengo que ver bien la hoja de datos)
							010     S8252
							011	S53    (no lo agregue al attiny, pero tiene que ser un codigo distinto)
			i: inverted sck	 		0:      el sck no esta invertido. En AVR, S82553, S52, S53
							1:	el sck esta invertido. En8253

	 wValue high: no se usa.

wIndex:  Se divide en parte alta y baja

	wIndex low:     Es el comando a ejecutar en modo rafaga. Puede ser read/write flash/eprom. El micro lo necesita para armar el comando cuando le lleguen los 
			bytes con los datos.

	wIndex high:	Si la funcion a realizar es un WRITE, el attiny apenas hace la escritura, hace una lectura del dato para ver si se terminó de escribir.
			En este byte se le pone el byte comando de lectura. Ademas permite ganar espacio en el Attiny al no tener que hacer la XOR en el micro (en AVR
			y en todos los modelos aca usados, pasar de WRITE a READ, es hacer una xor)y se gana generalidad.

Ejemplo de la trama:

	val = TAMANIO_8253 + MICRO_8253 + INVERTED_SCK_MASK;

	usb_control(USBTINY_CONFIGURE,val,ind);


\section{Defines + enums}

Enum de las tramas:

enum
{
	// Generic requests
	USBTINY_ECHO,		// echo test
	USBTINY_READ,		// read byte
	USBTINY_WRITE,		// write byte
	USBTINY_CLR,		// clear bit 
	USBTINY_SET,		// set bit
	// Programming requests
	USBTINY_POWERUP,	// apply power (wValue:SCK-period, wIndex:RESET)
	USBTINY_POWERDOWN,	// remove power from chip
	USBTINY_SPI,		// issue SPI command (wValue:c1c0, wIndex:c3c2)
	USBTINY_POLL_BYTES,	// set poll bytes for write (wValue:p1p2)
	USBTINY_FLASH_READ,	// read flash (wIndex:address)
	USBTINY_FLASH_WRITE,	// write flash (wIndex:address, wValue:timeout)
	USBTINY_EEPROM_READ,	// read eeprom (wIndex:address)
	USBTINY_EEPROM_WRITE,	// write eeprom (wIndex:address, wValue:timeout)
	USBTINY_DDRWRITE,        // set port direction
	USBTINY_SPI1,            // a single SPI command
	USBTINY_CONFIGURE	 // if command sent, use the inverted sck for 8253. (wValue = status, wIndex = comando)
};

Mascaras status:

#define TAMANIO_MASK		0x07
#define TAMANIO_AVR		0x04
#define TAMANIO_8253		0x04
#define TAMANIO_8252		0x03

#define MICRO_S51_MASK		0x38
#define MICRO_AVR		0x00
#define MICRO_8253		0x08 //incluye al S52 (ojo con el SCK)
#define	MICRO_8252		0x10 //incluye al S53



#define INVERTED_SCK_MASK	0x40	
#define AVR_SCK			0x00



\section{Secuencia}

Pasos a seguir

1-
Mandar trama de powerup al tiny: Configura los puertos y prende el led de programacion.
los argumentos son 

2- Mandar trama CONFIGURE

Se pondra el status en lo que corresponda, el wIndex no tiene importancia al momento. 
Es necesario mandar esta trama ahora para que el micro tenga bien la cantidad de bytes a mandar por el SPI, sino nunca podra entener la trama de ENABLE.

\end{document}
